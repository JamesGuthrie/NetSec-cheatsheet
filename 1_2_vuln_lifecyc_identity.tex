\section{Vulnerability lifecycle \& security}

\paragraph{Why is network security an issue?}
\begin{itemize}
\item Economy and life more and more depends on the internet
\item Distributed information systems have become critical infrastructures
\item Open systems: physical security is no longer sufficient
\item Insecurity driven by organized crime.
\item Huge and fast growing internet user base (2 billion), increasing risk
\end{itemize}

\paragraph{Security has become critical:}
\begin{itemize}
\item Security is one of the hidden building blocks of the internet.
\item The limits of security will become the limits of the internet
\item Growing online business attracts more attackers, attackers increase
the cost of doing business online
\item But the opportunities of being on the internet far outweigh the risks
\end{itemize}
\emph{Several decades of security problems, attacker knowledge
required has gone down dramatically (script kiddies)}

\paragraph{Asymmetric threat \& leverage:}
Attacker tries a few
exploits on a few systems, defender must
secure all systems against all exploits. New
opportunities for leverage all the time.

\textbf{Leverage:} You may reach 100 million
potential subjects (customers, or victims).
We can't count on previous constraints (e.g.
travel cost, cost of physical shipment) to limit
the effectiveness of an attacker. Scary
aspect of modern technology.

\paragraph{Attacker Motivation:}
\begin{itemize}
\item Ego: show off, impress
\item Revenge, destruction, creation of fear
\item Criminal intent: blackmail, fraud, e-banking
\item Acquisition of computing and network
resources: Botnets
\item Acquisition of sensitive information:
espionage
\end{itemize}

\paragraph{What can attackers do?}
\begin{itemize}
\item Attack flow of information
\item Attack services to achieve DoS
\item Attack infrastructure (DNS, ARP)
\item Gain access to services
\item Infiltrate security protocols/processes
\item Change system functions
\item Modify web pages
\item Hijack user sessions
\item Assume false identity
\item Use social engineering to establish trust
\item Break crypto
\end{itemize}

\textbf{Where are the attack targets?} Hackers attack us where we sit: Client-side
attacks dominate. Attacks of all shapes and
sizes. Data stored on end-points is often
most valuable and the least protected.

\paragraph{Security Concept - CIA triad:} The CIA triad is given by

\begin{tabular}{p{0.3\linewidth}p{0.3\linewidth}p{0.3\linewidth}}
\textbf{C}onfidentiality & \textbf{I}ntegrity & \textbf{A}vailability \\
\end{tabular}
(and more: Authenticity, Accountability, non
repudiation, privacy)

\paragraph{Attack Classification}
\begin{itemize}
\item Passive attacks: \emph{Confidentiality:} Compromise content, traffic analysis
\item Active attacks: \emph{Availability:} DoS, \\ \emph{Integrity and Authenticity:} Modification, Fabrication, Replay
\end{itemize}

\paragraph{Communication channel mode}
Not authentic/authentic, Not confidential/confidential

\emph{Secure = authentic and confidential}

Secure communication using insecure channel: Attacker has full
access to physical channel, knows all mechanisms and protocols,
does not know any secret keys.

Security can be implemented at different OSI layers

\paragraph{Insecurity Landscape}
Complexity is our worst enemy: Security of a
system = security of its weakest link

Complexity increases fast (more features,
code size, connectivity, embedded devices)

\emph{We are not willing to sacrifice features for
security.}

\paragraph{Vulnerabilities} 
A \emph{security vulnerability} refers to a weakness
in a system allowing an attacker to violate
CIA of the system.

There may be disagreement in concrete
cases: ``it's a feature not a vulnerability''.

The security landscape is determined by
vulnerabilities.

CVE = standardized vulnerability names
CVE = Common Vulnerability Exposure:
aims to standardize the names for all
publicly known vulnerabilities and security
exposures, de facto standard (ex.
CVE-2007-0943) (38000+ since 1995)

Every relevant security issue will get a CVE
number assigned.

\paragraph{Vulnerability Lifecycle}
Creation $\rightarrow$ discovery $\rightarrow$ exploit $\rightarrow$ disclosure $\rightarrow$ patch available $\rightarrow$ patch installed

\textbf{Risk time intervals:} pre-disclosure risk $\rightarrow$ post-disclosure risk $\rightarrow$ post-patch risk $\rightarrow$ day of disclosure = zero day

\begin{itemize}
\item pre-disclosure risk: time from discovery to disclosure, only few know
\item post-disclosure risk: time from disclosure to patch, public aware of
risk but still in danger
\item post-patch risk: time from patch availability to patch installation
\end{itemize}

\pgra{Zero-day vulnerability:} A zero-day (zero-hour, day zero) vulnerability is a vulnerability in a computer application that has been previously unknown, allowing attacks which take advantage of this lack of ``prior knowledge''. There are zero days between the time the vulnerability is discovered an the first attack. The developers have had zero days to address and patch the vulnerability.

\paragraph{From discovery to disclosure}
\begin{itemize}
\item  Less than zero day: Vulnerabilities not yet
known to the public are systematically used
by cybercriminals, government agencies,...
\item There is a market for new vulnerabilities:
Zero Day initiative of TippingPoint,
iDefense, black market, pricing from 1000
to 75000\$
\end{itemize}
Exploit availability jumps from 15\% to 80\%
at disclosure. At disclosure, less than 50\% of
vulnerabilities have a patch. The bad are
consistently faster than the good

\paragraph{Risk analysis:}
\begin{enumerate}
\item What assets are we trying to protect?
\item What are the risks of those assets?
\item How well does the security solution handle those risks
\item What other risks does the solution cause?
\item What costs and trade-offs does the
solution have?
\end{enumerate}
Finally: Is the trade-off worth it?

\paragraph{Risk management}
Security is relative (many risks possible). Options: avoid risk (give up), decrease risk
(technology), transfer risk (insurance),
accept risk. Take risk at the right place.

\textbf{Security is a trade-off}: No perfect security, we can have as much security as we want, we make decisions
every day about these trade-offs.

\section{Identity \& Authentication}
\paragraph{Definition:} An identity specifies a principal (unique entity). Examples: Individuals (e.g. person name), physical objects (e.g.
computer - network address, router, smart cart, airplanes), logical
objects (software - application name/version, URL), groups of
principals

\paragraph{Identity Theft:} Identity theft is a crime in which impostors obtain key
pieces of \emph{personally identifying information (PII)} such as social
security numbers (SSN), birth date, etc. and use them for their own
personal gain or in order to do harm.

\pgra{Prevention and Reaction:} Don't make PII available to untrusted
people. React fast if you suspect theft of your identity. Buy an identity
theft protection insurance if you are exposed. (fastest growing crime
in US, every 3 seconds an identity is stolen. 43\% of victims knew the
perpetrator)

\textbf{Variants:} Financial Identity Theft, Criminal Identity Theft, Identity
Cloning, Business/Commercial Identity Theft

\begin{tabular}{p{0.28\linewidth}p{0.62\linewidth}}
\textbf{Access control:} & The process of exerting control over who can interact with a resource \\
\hline
\hline
\textbf{Identification} & The process of establishing who someone or something claims to be for example by providing the system with a username \\
\hline
\textbf{Authentication} & The process of confirming a claimed identity, for example by verifying the user's password in the system \\
\hline
\textbf{Authorization} & The process of regulating what a user can do in the system, for example whether he can send a print job or not \\
\end{tabular}

\paragraph{Authentication:} Process of verifying an identity claim of an entity. Variants:
\begin{itemize}
\item Something an entity knows (PW, PIN)
\item Something an entity has (key card)
\item Something an entity is (biometric attributes) 
\item (a location where an entity is)
\item an ability an entity has (signing)
\end{itemize}

\begin{tabular}{|p{0.6\linewidth}|p{0.05\linewidth}|p{0.05\linewidth}|p{0.1\linewidth}|}
\multicolumn{4}{l}{\textbf{Authentication solutions}} \\
\hline
Type & Is & Has & Knows \\
\hline
\hline
\multicolumn{4}{l}{\textit{Weak authentication}} \\
\hline
iTan (list with label/alphanumeric types) & & X & \\
\hline
mTan (access to second communication channel) & & X & \\
\hline
Psylock (way of typing on a keyboard) & X & & \\
\hline
PassFaces (set of five faces) & & & X \\
\hline
SecLookOn (set or rules) & & & X \\
\hline
\multicolumn{4}{l}{\textit{Strong authentication}} \\
\hline
Internet Passport (fingerprint, device with crypt. keys) & X & X & \\
\hline
Digital Signature (smart card, PIN) & & X & X \\
\hline
SuisseID (smartcard or USB device, PIN) & & X & X \\
\hline
\end{tabular}

\paragraph{Authentication Techniques}
Biometrics: Fingerprint, retina scan, etc.\\
Crossover Error Rate: The lower, the better (where false reject rate = false accept rate) \\
Passwords: multi vs. single use (tokens) \\
Cryptographic keys: passphrase to unlock keys, memory/smart cards \\
\emph{Weak authentication:} means checking only
one authentication criteria \\
\emph{Strong authentication:} means checking two or more authentication criteria. (ex. ATM machines, savings account, digital signature, access to vault)

\paragraph{Authentication by a trusted third party}
Verification: compare image and/or signature with holder of passport, check document authenticity (security features) \\
Digital counterpart: Digital certificates by CA

\paragraph{OpenID}
Open standard for decentralized user authentication. OpenID allows
you to use an existing account to sign in to multiple websites,
without needing to create new passwords. Your password is only
given to your identity provider and that provider confirms identity to
the websites you visit.

(1. User enters OpenID, 2. Discovery, 3.
Authentication, 4. Approval, 4.a Change Attributes, 5. Send
Attributes, 6. Validation)

\paragraph{OAuth}
\emph{Traditional:} A web user (resource owner) grants a printing service
(client) access to her protected photos stored at a photo sharing
service (resource server) by sharing her password with the printing
service. The problems:
\begin{itemize}
 \item Third-party applications are required to store the password for
future use
 \item servers are required to support PW authentication
 \item third-party applications gain overly broad access
  \item resource owners cannot revoke access without changing PW
\end{itemize}

\emph{OAuth:}

\begin{tabular}{c|ccc|c}
  Client  &  & Auth. request & $\rightarrow$ & Resource \\ 
  & $\leftarrow$ & Auth. grant &  & Owner \\ 
 &  & Auth. grant & $\rightarrow$ & Authorization \\ 
   & $\leftarrow$ & Access token &  & Server \\ 
  &  & Access token & $\rightarrow$ & Resource \\ 
 & $\leftarrow$ & Protected &  & Server \\ 
 &  & Resource &  &  \\ 
\end{tabular} 

\textbf{Anonymity}

\begin{tabular}{lcr}
\emph{Authenticated} & & \emph{Anonymous} \\
Identity & Pseudonymity & Anonymity \\
\hline
Public & Non-public & Unlinkable \\
pseudonyms & pseudonyms & pseudonyms \\
\end{tabular}

\textbf{Anonymous networks:} Onion routing, Mixnets. \\
\textbf{Attacks against anonymity:} Traceback, Collusion, Traffic analysis, logging

\paragraph{3D Secure (3DS)}
Protocol used to authenticate online card transactions. Global framework. Reduces operational expense. No specialized cardholder SW or HW. Is extensible into emerging channels (mobile). Centralized archive of payment authentications.

Three Domains (3D): Issuer Domain, Interoperability Domain, Acquirer Domain

\textbf{Goal: less online fraud, but liability shift to customer side!}

\textbf{Problems:} pop-up blockers ($\to$ iframe), hidden security clues, activation during
shopping, liability shift, weak authentication, PW reset procedure not specified, privacy issues. Improvements: Some banks use per transaction SMS codes instead of PWs.

\paragraph{802.1x}
Client-server based access control protocol (WLAN). Standard data link layer protocol. 
Creates two virtual access points at each port.(controlled/uncontrolled) 1. User activates link 2. Authentication server sends an authentication challenge 3. User responds 4a. Authentication server checks response 4b Switch opens controlled port \\
 Supplicant, Authenticator, Auth. Server

\paragraph{802.1x EAP}
Extensible Authentication Protocol, identification framework, supports multiple
methods (MD5, OTP, TLS) \\
Access control = Authentication, Authorization

\textbf{802.1x Benefits/Limitations:} + standard-based technology, + extends authentication, + controls exercised at link layer, + interoperates in wired, wireless scenarios\\
- authenticator identification, - man-in-the-middle

