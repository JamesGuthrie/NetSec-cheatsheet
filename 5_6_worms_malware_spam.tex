\section{Malware \& Worms}

\paragraph{Computer virus:} a computer program, which
\begin{itemize}
\item infects other programs by embedding itself (infection),
\item propagates (propagation) and possibly mutates (mutation,
polymorphism) during infection,
\item possibly upon a trigger event executes malicious code
(daemon, malcode).
\end{itemize}

\subsection{Malware}

\paragraph{Definition:} software intentionally included or inserted in a computer system for a harmful purpose. Types of malware:
\begin{itemize}
\item Trojan (software that seems useful but also contains hidden functionality to undermine system security)
\item Bot (malicious software agent running on a compromised machine. Executes commands from botnet operator. Listens to \emph{command and control center})
\item Rootkit (activate during boot; hide from operating system)
\item Keylogger (records keys pressed to steal information)
\item Dialer (misuse Internet dial-up modem)
\item Ransomware (encrypt data; extort money for encryption key) 
\item Fake Anti-``malware removal'' tools (keep system infected)
\end{itemize}

\paragraph{Malware on portable storage} Propagates via ``sneakernet'', e.g. people physically carrying the virus from $A$ to $B$. Variants: (1) boot code virus (code executed on start up), (2) host program infection (virus embeds itself into executable files, documents, etc.)

\paragraph{Why is malware a problem?} unauthorized use of system/network resources,
sabotage, espionage (data leakage, watch victim),
lowering of system security, potential effect on
network: propagation activity causes (too much) network traffic (bandwidth exhaustion), botnet expansion, enabler of follow-up illicit activity (spam, etc.)

\emph{Cyber attacks under top 5 global risks in 2012!}

\paragraph{Malware Detection:} Antivirus software \emph{deployed at} (a) endpoint (check processes and settings, scan files), (b) network (scan web traffic at web proxies, scan emails and email servers), \\
using the following \emph{detection methods:} (i) signature-based (malware signature pattern matching), (ii) behavior-based (heuristics, statistics to detect malicious behavior of a software)

\paragraph{Antivirus software evaluation:} good av-software has to fulfil the following criteria:
\begin{itemize}
\item Detection (reactive/proactive)
\item Support of compressed files and streams
\item Self protection (rootkits)
\item Cleanup of infected system
\item High true positive rate, low false positive rate
\end{itemize}

\paragraph{Challenges of detection approaches:} Usually one method does not suffice. Most advanced av-softwares use a combination (50/50) of both. Best: Kaspersky

\begin{tabular}{p{0.2\linewidth}p{0.7\linewidth}}
Approach & Challenge \\
\hline
\hline
Signature-based & Postmortem, based on a known exploit, always behind (hours, days, weeks), signature database grows forever (thousands of new signatures per day), must keep old signatures, matching against all signatures (computationally expensive) \\
\hline
Behavior-based & Requires a behavior model $\to$ define ``normal'' ‏(ground‏truth, ‏baseline), anomalous (unusual registry/network access patterns, unusual access(es) to systemfiles), alerting (high false positive rate will confuse the user!) \\
\end{tabular} 

\paragraph{APT -- Advanced Persistent Threat:} sophisticated stealth customized attack on selected high value targets.\\
\textbf{Examples:} Aurora (backdoor, sleeping cells of malware inside corp., 2009), Stuxnet (manipulate rotation speed of uranium centrifuges in Iran, 2010), Staatstrojaner (German/Swiss government spyware on suspected criminals, 2011)

\begin{tabular}{p{0.2\linewidth}p{0.7\linewidth}}
Advanced & Sophisticated attack techniques, e.g. zero
day attacks, knowledge of source code, wiretapping, social
engineering, and highly targeted deployment \\
\hline
Persistent & Attack might be launched anytime, well hidden, organization might already be infected, hard to detect \\
\hline
Threat & Attackers have a specific objective and are skilled,
motivated, organized and often well funded. A serious risk to
the ‏targeted ‏organization’s‏ data ‏and ‏mission. \\
\end{tabular}

\paragraph{Antivirus detection circumvention:}
\begin{itemize}
\item Polymorphism (every replication is a mutation $\to$ makes antivirus signature based approaches inefficient)
\item Code obfuscation
\item Disguise actual purpose, make analysis hard
\item Encryption of code and messages
\item Code changes (icing) to prevent disassembly
\item Virtual machine/sandbox detection (won’t run malicious code in vm (or block execution at all))
\item Bootstrapping/multi-stage worms (malware spread over multiple downloads)
\end{itemize}

\subsection{Worms}

\paragraph{Definition:} self-contained software that can replicate itself from computer to computer across network connections; may be activated to perform some unwanted operation and continue propagating.

\paragraph{Propagation techniques:} A typical worm:\\ (1) select a target host address $\to$ (2) contact target host $\to$ (3) test host for exploitable vulnerabilities $\to$ (4) exploit vulnerabilities $\to$ (5) install copy of itself on target host $\to$ (6) execute copy on target host $\to$ (7) go to (1).

\paragraph{Exploitable vulnerabilities:} In server: web server/FTP server, client application: IE, p2p, network layer: VPN software, user: trick the user

\paragraph{How to find host?} Sparsely populated address spaces (Email, IPv4, not IPv6), use local topological knowledge (ARP caches, contact lists, IPv4 subnet) \\
\emph{Random target selection:} tree-like (robust, fast), one-to-all (greedy), one-to-one (not robust, slow)

\paragraph{Trigger Events \& User-in-the-loop} Some malware gets activated by a trigger event. 
\begin{itemize}
\item typically limits spreading speed
\item usually only one event per target needed
\item usually automatic propagation afterwards
\end{itemize}
\textbf{User interaction examples:} Email virus $\to$ open an attachment, click on a link in an IM, execute a file shared in a p2p network, boot from floppy disk

\paragraph{Propagation speed:} \emph{flash worm} (fast, small, precomputed targets, sub-minute saturation time) vs. \emph{slow worm} stealthy, minimally intrusive, undetected for a (very) long time, hard to detect. \\
Propagation speed depends on:  Scan rate, vulnerable population, system compromise delay, worm transfer speed and other factors.

\paragraph{SIS-model:} taken from biological virus epidemiology. Most of the time a one-to-one mapping from bio to computer is possible, however: (a) bio-viruses spread much slower than computer viruses, (b) computers are not self-healing (lack of immune system). \\
Three worm-spreading stages: pre-outbreak, free spreading, clean-up \\
Two discrete stages in model: \textbf{S}usceptible and \textbf{I}nfected, dynamics $S \to I \to S$, total of $N$ nodes in network

\begin{tabular}{p{0.2\linewidth}p{0.7\linewidth}}
Parameter & Explanation \\
\hline
\hline
$\beta \in \left[0, 1\right]$ & Infection probability of an $S$-node along each edge that connects it to an $I$-node \\
$\delta \in \left[0, 1\right]$ & Cure prob. of an $I$-node. Transforms an $I$-node to an $S$-node \\
$k \in \mathbb{N}$ & number of outgoing edges at each node ($\approx const$)  \\
$\rho(t) \in \left[0, 1\right]$ & Fraction of connected nodes \\
\end{tabular}

The dynamics of $\rho(t)$ are modeled by a first order ODE:
\dm{
\dot{\rho}(t) = \frac{d\rho(t)}{dt} = \underthebrace{\text{\# of }S \text{-node infections}}{\beta k (1 - \rho(t)) \rho(t)} - \underthebrace{\text{\# of spont. cures}}{\delta \rho(t)}
}

Epidemic threshold
\dm{
\lim_{t \to \infty} \rho(t) = \begin{cases} 0, & \lambda > 1 \,(\text{exp. decay, no epidemic}) \\ 1- \lambda, & \lambda < 1 \,(\text{exp. growth, epidemic}) \end{cases}
}
with inverse spreading rate $\lambda := \delta/(\beta k)$. 

\paragraph{(In-)Famous worms} \emph{Morris (1988):} First internet worm, 10\% of hosts infected, DoS \\
\emph{SQL Slammer (2003):} Entire worm fits inside a single UDP packet, DoS \\
\emph{Blaster (2003):} Exploit of 135/TCP, multi-stage propagation \\
\emph{Sobig.F (2003):} Email worm, worm in attachment, own SMTP engine\\
\emph{Witty (2004):} Attacked windows firewall product, erased random data

\paragraph{Detection of worm network impact:} scanning activity is primary network impact, high ARP activity, high volume of ``ICMP dest. unreachable'' (scan
misses), more network traffic and flow \\
\emph{Detection by:} Network level: IDS/IPS, scan detectors \\
Backbone level: Black holes, network activity anomaly detection

\paragraph{Countermeasures} Worms are hard to get rid of. No definitive solution exists, just attempts to reduce risk:
\begin{itemize}
\item[+] Secure end system (patching, firewall, NAT, IPS, anti-vir), educate users
\item[+] Appl. whitelisting (only run trusted, white-listed software (can be flawed however))
\item[+] Intranet: closed user group for Internet services, protect internal appl. 
\item[-] Monocultures (e.g. Windows, Mac OS X, etc.) targeted first, have slow patching
\item[-] Virus-scanners often too late
\end{itemize}

\paragraph{Email worm characteristics} Typical mechanism: emails itself to users as attachment, opening attachment launches virus, new targets acquired from address books. Vulnerability is based on a design/UI problem of email clients and social engineering.

\section{SPAM}

\paragraph{Email:} Email allows fast, low cost, worldwide, multi format communication. The email protocols defined in RFC 821 and 822 allowed researchers to exchange messages across the unreliable but totally trusted early internet.

\paragraph{Definition:} Spam is junk-mail, unsolicited email, often commercial and frequently sent in bulk.\\
UCE = unsolicited commercial email \\
UBE = unsolicited bulk email (non commercial email sent out in masses)

100 known spam operations are responsible for 80\% of spam worldwide, Shutdown of ISP McColo in 2008 reduced global spam by up to 75\%

\paragraph{Why is there spam?}
\begin{itemize}
\item Direct sales business (some people actually buy spam advertised products, cheap advertising channel)
\item Internet underground business, operators rent out botnets for sending spam (1 million spam emails sent for 250-700\$)
\item Malware distribution
\end{itemize}

\paragraph{What problems does spam cause?} 60-95\% of today's email traffic is spam. \\
Annual global footprint equivalent to 3 million vehicles on the road annually \\
Problems caused: misuse of technical resources, waste of time and money, DoS, increased IT cost, malware in spam

\paragraph{SPAM sending techniques:}
\begin{itemize}
\item Spam sent from ``borrowed'' email accounts
\item Sending spam from short-term officially registered domains
\item Use of fake sender domains (no DNS query) 
\item Sending spam from botnet machines
\end{itemize}

\paragraph{Anti-Spam techniques:} Fuzziness of the problem: what is spam? \\
Generic Solution: Mailfilter\\
What to do with spam emails?: delete, mark, quarantine, tag\\
Spam score: what? who? where? how?

\paragraph{Spam examples:}
\begin{itemize}
\item LinkedIn Spam: notifications waiting
\item Penny Stocks Spam: spam about hot stocks
\item Nigerian Money Scam Spam: you get to keep 10\% if you help transfer money
\item Work from home: Mule recruitment spam: regional assistants, keep 10\%, money laundering, illegal activities
\item Phishing Emails Spam
\item Rock-Phishing Email Spam: use URLs with legitimate looking second level domain (sess99.mybank.com)
\item Image spam: spam message as text letters in image
\end{itemize}

\paragraph{How did they get my email address?}
\begin{itemize}
\item directory harvest attack against mail server
\item email address crawlers (websites and forums)
\item malware (collects locally stored email addresses)
\item harvesting email addresses from databases (hacked online shops, etc.)
\end{itemize}

\paragraph{Filtering at SMTP time:} SMTP message format:
\begin{enumerate}
\item {\tt HELO} $<$sender's hostname$>$
\item {\tt MAIL FROM}: $<$sender's email address$>$
\item {\tt RCPT TO}: $<$recipient's email address(es)$>$
\item {\tt DATA}
\item $<$header of email$>$
\item {\tt <CR><LF>}
\item $<$body of email$>$
\item {\tt <CR><LF>.<CR><LF>}
\end{enumerate}

Response codes: 2xx (OK), 3xx (need more data), 4xx (temporary failure)
\begin{enumerate}
\item Client restrictions: Deny after learning client's (sender's) IP 
\item Helo restrictions: Deny at {\tt HELO/EHLO}
\item Sender restrictions: Deny at {\tt MAIL FROM}
\item Recipient restrictions: Deny at {\tt RCPT TO}
\item Data restrictions: Deny at data
\item End of data restrictions: Deny at end of data section
\item Etrn restrictions: Deny at {\tt ETRN} (extended turn)
\end{enumerate}

\textit{Advantage:} Cheap and fast; no long email body received or processed \\
\textit{Disadvantage:} Not too much information available to decide if spam

\paragraph{Filtering SMTP traffic at Firewall:} FW blocks outgoing SMTP port for all hosts except for known internal mail server

\paragraph{Heuristic Content Filtering:} Run algorithms that look for a probable spam feature: weird use of fonts, tiled images, strange URLs 

\paragraph{Statistical Content Filtering:} Naive Bayes Filter:
\dm{
 P(spam|words) = \frac{P(words|spam) P(spam)}{P(words)} 
 }

\paragraph{Traffic Based Filtering:} Distributed checksum clearinghouse (DCC) counts number of same email messages in the past 

\paragraph{Image Spam Filtering:} Emails with image attachments or inline images. Apply OCR (Optical Character Recognition) to image to extract text

\paragraph{Text Filtering caveats:} Normalize content before filtering: HTML formatting, white text on white background, tiny letters, misspellings in text (defeats Bayes), URL with redirects to other URL

\paragraph{White-/Grey-/Blacklisting:} \emph{Whitelisting:} allow emails matching non-spam criteria \\
\textit{Greylisting:} defer initial email but accept follow up \\
\textit{Blacklisting:} reject emails matching spam criteria

\paragraph{How blacklists work:} IP address lookup in DNS blacklists. Each blacklisted IPv4 address or domain name can be queried from the DNS with \\
{\tt <IP-address>.<DNS-blacklist-domain>} or {\tt <domain>.<...>}

\paragraph{How realtime blacklists differ:}
\begin{itemize}
\item Operator
\item Goal: Find hosts that have done things that proper SMTP servers don't do
\item Nomination: manual, users, testers 
\item Listing lifetime: 5 [s] -20 [min], temporary 
\item Cost: mostly free
\end{itemize}

\paragraph{Blacklist caveats:} Blacklist evasion: spammer use stolen email accounts, Denial of Service: If your mail host gets on a blacklist your mails might no longer arrive

\paragraph{DKIM:} Domain Keys Identified Mail \\
Sign a hash of message (some header fields and full content) with private key, rcpt gets public key from DNS text record and verifies signature $\to$ Message integrity, sender auth. 

\paragraph{Sender Policy Framework -- SPF:} validates HELO domain and validates the MAIL FROM address. Domain owners publish lists with IP addresses that are authorized to use their domain

\paragraph{SenderID:} validates one of the message's address header fields
